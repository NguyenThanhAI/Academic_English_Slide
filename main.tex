%% This Beamer template is based on the one found here: https://github.com/sanhacheong/stanford-beamer-presentation, and edited to be used for Stanford ARM Lab

\documentclass[10pt]{beamer}
%\mode<presentation>{}

\usepackage{media9}
\usepackage{amssymb,amsmath,amsthm,enumerate}
\usepackage[utf8]{inputenc}
\usepackage{array}
\usepackage[parfill]{parskip}
\usepackage{graphicx,animate}
\usepackage{caption}
\usepackage{subcaption}
\usepackage{bm}
\usepackage{amsfonts,amscd}
\usepackage[]{units}
\usepackage{listings}
\usepackage{multicol}
\usepackage{multirow}
\usepackage{tcolorbox}
\usepackage{physics}
\usepackage{movie15}
% Enable colored hyperlinks
\hypersetup{colorlinks=true}

% The following three lines are for crossmarks & checkmarks
\usepackage{pifont}% http://ctan.org/pkg/pifont
\newcommand{\cmark}{\ding{51}}%
\newcommand{\xmark}{\ding{55}}%

% Numbered captions of tables, pictures, etc.
\setbeamertemplate{caption}[numbered]
\usepackage{media9} 
%\usepackage[superscript,biblabel]{cite}
\usepackage{algorithm2e}
\renewcommand{\thealgocf}{}

% Bibliography settings
%\usepackage[style=authoryear]{biblatex}
%\setbeamertemplate{bibliography item}{\insertbiblabel}
%
%\addbibresource{bibliography.bib}
\usepackage[natbib,backend=biber,style=ieee, sorting=ynt]{biblatex}
\bibliography{ref.bib}

% Glossary entries
\usepackage[acronym]{glossaries}
\newacronym{ML}{ML}{machine learning}
\newacronym{HRI}{HRI}{human-robot interactions}
\newacronym{RNN}{RNN}{Recurrent Neural Network}
\newacronym{LSTM}{LSTM}{Long Short-Term Memory}


\theoremstyle{remark}
\newtheorem*{remark}{Remark}
\theoremstyle{definition}

\newcommand{\empy}[1]{{\color{darkorange}\emph{#1}}}
\newcommand{\empr}[1]{{\color{cardinalred}\emph{#1}}}
\newcommand{\examplebox}[2]{
\begin{tcolorbox}[colframe=darkcardinal,colback=boxgray,title=#1]
#2
\end{tcolorbox}}

\usetheme{Stanford} 
\input{./style_files_stanford/my_beamer_defs.sty}
\logo{\includegraphics[height=0.5in]{logos/HUS-name.jpg}}

\makeatletter
\let\@@magyar@captionfix\relax
\makeatother

\title[Title]{Backpropagation in Training \\ Deep Neural Networks}


\begin{document}

\nocite{*}

\author[Author]{
	\begin{tabular}{c} 
	\Large
	Chi Thanh Nguyen \\
    \footnotesize \href{mailto:nguyenchithanh\_sdh21@hus.edu.vn}{nguyenchithanh\_sdh21@hus.edu.vn} \\
	\Large
    Duc Thinh Nguyen \\
     \footnotesize \href{mailto:nguyenducthinh\_sdh21@hus.edu.vn}{nguyenducthinh\_sdh21@hus.edu.vn}
\end{tabular}
\vspace{-4ex}}

\institute{
	\vskip 5pt
	\begin{figure}
		\centering
		\begin{subfigure}[t]{0.5\textwidth}
			\centering
			\includegraphics[height=0.75in]{logos/HUS-logo.jpg}
		\end{subfigure}%
		~ 
		\begin{subfigure}[t]{0.5\textwidth}
			\centering
			\includegraphics[height=0.75in]{logos/MIM-logo.png}
		\end{subfigure}
	\end{figure}
	\vskip 10pt	
	Vietnam National University \\
	Hanoi University of Science\\
	Faculty of Mathematics, Mechanics and Informatics
	\vskip 3pt
}

% \date{June 15, 2020}
%\date{\today}

\begin{noheadline}
\begin{frame} \maketitle \end{frame}
\end{noheadline}

\setbeamertemplate{itemize items}[default]
\setbeamertemplate{itemize subitem}[circle]

\begin{frame}{Content}

\begin{enumerate}
	\item Introduction
	\item Formulations
	\item Example  
\end{enumerate}

\end{frame}

\begin{frame}
\frametitle{Introduction}
	\begin{itemize}
	    \item Item 1
		\item Item 2
		\item Item 3
		\begin{itemize}
    		\item Subitem 1 
    		\item Subitem 2
		\end{itemize}
	\end{itemize}
	
	Citation 1 \cite{rumelhart1986learning}
\end{frame}

\begin{frame}{Materials and Methods}
\end{frame}

\begin{frame}{Results}
\end{frame}

\begin{frame}{Discussion}
\end{frame}

\begin{frame}{Conclusion}
\end{frame}

\begin{frame}{Acknowledgements}
\end{frame}

\begin{frame}[allowframebreaks]{References}
\printbibliography
\end{frame}

\end{document}